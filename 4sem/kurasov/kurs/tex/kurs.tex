%
%  untitled
%
%  Created by Михаил Костенчук on 2009-05-29.
%  Copyright (c) 2009 __MyCompanyName__. All rights reserved.
%
\documentclass[a4paper,11pt]{article}
\usepackage[utf8]{inputenc}
\RequirePackage[english,russian]{babel}
\usepackage{fullpage}
\usepackage{verbatim}
\usepackage{minitoc}
\usepackage{graphicx}
\usepackage{caption2}
\renewcommand{\captionlabeldelim}{.} 
\title{Простейший SOCKS сервер}
\author{М.И. Костенчук}

\date{06-06-2010}

\begin{document}

\maketitle

\tableofcontents

\section{Введение}

Целью работы является написание простейшего SOCKS-сервера, обрабатывающего TCP/IP соединение. 

\section{Протокол SOCKS 5}

Соединение с SOCKS-сервером происходит следующим образом:

\begin{enumerate}

  \item Клиент подключается, и посылает приветствие, которое включает перечень поддерживаемых методов аутентификации

  \item Сервер выбирает из них один (или посылает ответ о неудаче запроса, если ни один из предложенных методов не приемлем)

  \item В зависимости от выбранного метода, между клиентом и сервером может пройти некоторое количество сообщений

  \item Клиент посылает запрос на соединение специального вида

  \item Сервер отвечает аналогичным образом

\end{enumerate}

Рассмотрим соединение подробней.

Первый запрос от клиента выглядит следующим образом:

\begin{tabular}{|c|c|c|}
  \hline  
  VER & NMethods & Methods \\
  \hline  
  1 & 1 & 1-255  \\  
  \hline
\end{tabular}

Где первая строка это названия полей, а вторая их размер:

\begin{itemize}
  
  \item VER~--- версия протокола. В нашем случае это поле равно 0x05.
  
  \item NMethods~--- Количество методов аутентификации.
  
  \item Methods~--- Методы аутентификации:
  
  \begin{itemize}
    
    \item 0x00~--- аутентификация не требуется
    
    \item 0x01~--- GSSAPI
    
    \item 0x02~--- USERNAME/PASSWORD (см. RFC 1929)
    
    \item 0x03 до 0x7F~--- зарезервировано IANA
    
    \item 0x80 до 0xFE~--- предназначено для частных методов
    
    \item 0xFF~--- нет применимых методов
    
  \end{itemize}
  
\end{itemize}

Сервер выбирает один из предложенных методов и посылает ответ в следующем виде:

\begin{tabular}{|c|c|}  
  \hline  
  VER & Method \\
  \hline  
  1 & 1 \\  
  \hline
\end{tabular}

Затем клиент и сервер начинают аутентификацию выбранным способом. Реализация аутентификации выходит за рамки данной работы, поэтому не будет рассмотрено. 

После успешной аутентификации клиент посылает серверу запрос вида:

\begin{tabular}{|c|c|c|c|c|c|}  
  \hline  
  VER & CMD & RSV & AType & DST.Addr & DST.Port \\
  \hline  
  1 & 1 & 1 & 1 & переменное & 2\\  
  \hline
\end{tabular}

\begin{itemize}

  \item VER~--- Версия протокола

  \item CMD~--- Тип запроса: 

  \begin{itemize}

    \item 0x01~--- Connect

    \item 0x02~--- Bind

    \item 0x03~--- UDP Associate

  \end{itemize}

  \item AType~--- Тип адреса хоста:

  \begin{itemize}

    \item 0x01~--- IPv4

    \item 0x03~--- Имя домена 

    \item 0x04~--- IPv6

  \end{itemize}

  \item DST.Addr~--- Адрес хоста в виде, указанном в AType:

  \item DST.Port~--- Порт хоста

\end{itemize}

Сервер отправляет ответ вида:

\begin{tabular}{|c|c|c|c|c|c|}  
  \hline  
  VER & REP & RSV & AType & BND.Addr & BND.Port \\
  \hline  
  1 & 1 & 1 & 1 & переменное & 2\\  
  \hline
\end{tabular}

\begin{itemize}

  \item VER~--- Версия протокола

  \item REP~--- Код ответа: 

  \begin{itemize}
    
    \item 0x00~--- Успешноый

    \item 0x01 до 0x08~--- Различные ошибки

  \end{itemize}

  \item AType~--- Тип последующего адреса (аналогично запросу):

  \item DST.Addr~--- Выданный сервером адрес

  \item DST.Port~--- Выданный сервером порт

\end{itemize}

Если сервер ответил 0x00, значит значит соединение прошло успешно и можно отправлять данные.

\section{Итог}

Полностью реализовано соединение с клиентом. Обмен данными происходит следующим образом: Сервер ждёт сообщения от клиента, после его приёма полностью пересылает удалённому хосту, и пересылает ответ обратно клиенту. Для проверки реализован собственный простейший socks-клиент.

\begin{thebibliography}{text} 

\bibitem{1} \label{bib_1} \verb+http://ru.wikipedia.org/wiki/SOCKS+~--- Википедия про SOCKS.

\bibitem{2} \label{bib_2} \verb+http://rfc2.ru/1928.rfc+~--- RFC 1928~--- протокол SOCKS5. 
	
\end{thebibliography} 

\end{document}










